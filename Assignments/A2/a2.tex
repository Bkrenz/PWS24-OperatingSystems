\documentclass[12pt]{article}

\title{Assignment 2}
\author{G3 - Emily Bolles, Tanner Kirsch, Robert Krency}
\date{}

\usepackage{subfiles}

\usepackage{tikz}
%Tikz Library
\usetikzlibrary{angles, quotes, intersections}



\usepackage{pgfplots}
\usepgfplotslibrary{polar}

%Notation
\usepackage{physics}
\usepackage{bm}

%AMS
\usepackage{amsmath}
\usepackage{amsfonts}
\usepackage{amssymb}
\usepackage{amsthm}

%Stuff
\usepackage{graphicx}
\usepackage{tabularx}
\usepackage{multicol}
\usepackage{algpseudocode}
\usepackage{algorithm}
\usepackage{enumitem}

\usepackage{setspace}
\onehalfspacing

% Geometry 
\usepackage{geometry}
\geometry{letterpaper, left=1in, top=1in, right=1in, bottom=1in}

\setlength\columnsep{20pt}

% Fancy Header
\usepackage{fancyhdr, lastpage}
\fancypagestyle{plain}{
  \fancyhf{}    % Clear header/footer
  \fancyhead[L]{ CMSC-4000 - Operating Systems}
  \fancyhead[R]{PennWest S24}
  \fancyfoot[C]{Page \thepage\ of \pageref{LastPage}}
}
\pagestyle{plain}


% Macros
\newcommand{\definition}[1]{\underline{\textbf{#1}}}

\newenvironment{rcases}
  {\left.\begin{aligned}}
  {\end{aligned}\right\rbrace}

% Begin Document
\begin{document}

\maketitle

\section*{Q1}

A computer system has enough room to hold 6 processes in its main memory.
These processes are idle waiting for I/O 65\% of the time.
What is the utilization of the CPU?
Show your answer steps in detail.

With process count \(n=6\) and wait time probability of \(p=.65\):
 
\begin{align*}
    \textrm{CPU\_Util} & = 1 - p^n \\
     & = 1 - {(.65)}^6 \\
     & \approx 1 - 0.075 \approx 0.925
\end{align*}

This gives us a CPU Utilization factor of \(92.5\%\).


\section*{Q2}

The CPU Utilization of a computer system with a large memory size was \(40\% \) with executing 6 processes that have the same I/O characteristics.
How many processes from the same type should be executed to achieve around \(94\%\) CPU Utilization?
Show your answer steps in detail.

Using the formula for CPU Utilization based on process count \(n=6\) and CPU Utilization of \(40\%\), we can work out the I/O characteristic \(p\).

\begin{align*}
    0.40 & = 1 - {(p)}^6 \\
    {(p)}^6 & = 1 - 0.40 = 0.60 \\
    p & = {0.40}^\frac{1}{6} \approx 0.9184
\end{align*}

We have thus determined that the I/O characteristic is approximately \(91.84\%\) for the processes in this system.
We use the formula again then to determine the process count \(n\) such that we achieve approximately \(94\%\) CPU Utilization.

\begin{align*}
    0.94 & = 1 - {(0.9184)}^n \\
    {(0.9184)}^n & = 1 - 0.94 = 0.06 \\
    \ln{0.9184^n} & = \ln{0.06} \\
    n \times \ln{0.9184} & = \ln{0.06} \\
    n & = \dfrac{\ln{0.06}}{\ln{0.9184}} \\
    n & \approx 33
\end{align*}

Therefore, we know that this system would require \(33\) similar class processes to achieve \(94\%\) CPU Utilization.

\end{document}