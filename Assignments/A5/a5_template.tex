\documentclass[12pt]{article}

\title{Assignment 5}
\author{G3 - Emily Bolles, Tanner Kirsch, Robert Krency}
\date{}

\usepackage{graphicx}

\usepackage{setspace}
\onehalfspacing

% Geometry 
\usepackage{geometry}
\geometry{letterpaper, left=1in, top=1in, right=1in, bottom=1in}

\setlength\columnsep{20pt}

% Fancy Header
\usepackage{fancyhdr, lastpage}
\fancypagestyle{plain}{
  \fancyhf{}    % Clear header/footer
  \fancyhead[L]{ CMSC-4000 - Operating Systems}
  \fancyhead[R]{PennWest S24}
  \fancyfoot[C]{Page \thepage\ of \pageref{LastPage}}
}
\pagestyle{plain}


% Macros
\newcommand{\definition}[1]{\underline{\textbf{#1}}}

\newenvironment{rcases}
  {\left.\begin{aligned}}
  {\end{aligned}\right\rbrace}

% Begin Document
\begin{document}

\maketitle

\section{Results}

\subsection{Multiple Consumers}

The chart in Figure 1 shows a comparison of the time taken to complete the task by the number of Consumer threads running.
As expected, more consumers running concurrently finishes the task much more quickly.

\begin{figure}[h]
  \includegraphics[width=12cm]{Consumers.png}
  \centering
  \caption{Number of Consumer Threads}
\end{figure}

\subsection{Item Buffer Sizing}

The chart in Figure 2 here shows a comparison between utilizing a 1-Item Buffer and a 2-Item Buffer for the Producer to fill.
The 2-Item Buffer could theoretically protect against the Producer being unable to gain access to the Buffer for longer periods of time,
due to the Consumers gaining the lock consecutively.
In practice, there is little difference.
This is likely a combination of small sample size and quickly performing tasks. 
There could be improvements by separating the buffers such that threads would be able to access them independently. 

\begin{figure}[h]
  \includegraphics[width=12cm]{ItemBuffer.png}
  \centering
  \caption{1- vs 2- Item Buffer}
\end{figure}

\pagebreak
\section{Primes Found}

\begin{tabular}{c c c c c c}

\VAR{ primes }
\end{tabular}

\section{A: Producer-Consumer}

\begin{tabular}{ | r | c || r | c |}
  \hline
  \textbf{Test \#} & \textbf{Time (s)} & \textbf{Test \#} & \textbf{Time (s)} \\ \hline  
\VAR{results_a}
\end{tabular}

\pagebreak
\section{B: Multiple Consumers}

\begin{tabular}{ | r | c c c c |}
  \hline
  \textbf{Consumers} & 2 & 3 & 4 & 8 \\
\VAR{results_b}
\end{tabular}

\pagebreak
\section{C: Two-Item Buffer}

\begin{tabular}{ | r | c c c c |}
  \hline
  \textbf{Consumers} & 2 & 3 & 4 & 8 \\
\VAR{results_c}
\end{tabular}

\section{System Specs}

The developed program was executed on a system with the following specs:

\begin{center}
\begin{tabular}{| r | l |}
    \hline
    CPU & AMD Ryzen 5 5600X, 6 Cores \\ \hline
    Motherboard & Gigabyte X570 Aorus Ultra \\ \hline
    Memory & 32GB DDR4 RAM 4800 MHz \\ \hline
    GPU & AMD Radeon RX 6700XT \\ \hline
\end{tabular}
\end{center}

\end{document}